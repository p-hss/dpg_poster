%%% --> Braces and Annotations <--
% braces for operators (removes the spacing before and after the delimiters)
\newcommand{\opleft}[1]{\mathopen{}\left#1}
\newcommand{\opright}[1]{\right#1\mathclose{}}
% flexible brace command, #1=left delimiter, #2=right delimiter, #3=size (normal, auto, opauto, \[sizecmd]), #4=content
\newcommandx{\braces}[4]{%
\ifstrequal{#3}{normal}{#1#4#2}{%
\ifstrequal{#3}{auto}{\left#1#4\right#2}{%
\ifstrequal{#3}{opauto}{\opleft#1#4\opright#2}{%
#3#1#4#3#2}}}%
}
% annotation on the top of relation symbols
\newcommandx{\opannot}[3][3=\downarrow]{\stackrel{\mathclap{\substack{#1 \\ #3 \vspace{2pt}}}}{#2}}
% annotation in front of a line
\newcommandx{\lineannot}[3][3=\rightarrow]{\mathllap{\boxed{\text{\textsmaller{#1}}} #3} #2}
% annotation in front of a line (multiline)
\newcommandx{\multilineannot}[4][4=\rightarrow]{\mathllap{\boxed{\parbox{#1}{\RaggedRight\textsmaller{#2}}} #4} #3}
% annotation between two lines
\newcommand{\interannot}[1]{\boxed{\text{\textsmaller{#1}}}} 
% annotation between two lines (multiline)
\newcommand{\multiinterannot}[2][.5\textwidth]{\boxed{\parbox{#1}{\RaggedRight\textsmaller{#2}}}}

%%% --> Number Sets and Symbols <--
\newcommand{\N}{\mathbb{N}} % natural numbers
\newcommand{\Nzero}{\mathbb{N}_0} % natural numbers with zero
\newcommand{\Z}{\mathbb{Z}} % integer numbers
\newcommand{\Q}{\mathbb{Q}} % rational numbers
\newcommand{\R}{\mathbb{R}} % real numbers
\newcommand{\Rpos}{\mathbb{R}_{>0}} % positive real numbers
\newcommand{\C}{\mathbb{C}} % complex numbers
\newcommand{\K}{\mathbb{K}} % field (real or complex numbers)
\newcommand{\eps}{\varepsilon} % shortcut for epsilon
\renewcommand{\Re}{\operatorname{Re}}
\renewcommand{\Im}{\operatorname{Im}}

%%% --> Logic <--
\renewcommand{\iff}{\Leftrightarrow} % if and only if
\renewcommand{\implies}{\Rightarrow} % implies
\newcommand{\suchthat}[1][normal]{\ifstrequal{#1}{normal}{\mid}{#1|}} % seperator in sets (#1op = size)

%%% --> Sets and Topology <--
\newcommand{\setcompl}[1]{#1^c} % complement of a set
\newcommand{\cardinality}{\#} % cardinality of a set
\newcommand{\union}{\cup} % union
\newcommand{\disjunion}{\mathrel{\dot{\union}}} % disjoint union
\newcommand{\bigunion}{\bigcup} % big union
\newcommand{\bigdisjunion}{\mathop{\dot{\bigcup}}} % big disjoint union
\newcommand{\intersec}{\cap} % intersection
\newcommand{\bigintersec}{\bigcap} % big intersection
\newcommand{\boundary}[1]{\partial#1} % boundary of a set
\newcommand{\clos}[1]{\overline{#1}} % topological closure of a set
\newcommand{\interior}[1]{\mathring{#1}} % interior of a set
\newcommand{\dist}[2]{\operatorname{dist}(#1, #2)} % distance between two sets

%%% --> Analysis <--
\newcommandx{\intvcl}[3][1=normal]{\braces{[}{]}{#1}{#2, #3}} % closed interval (#1op=size, #2=left bound, #3=right bound)
\newcommandx{\intvop}[3][1=normal]{\braces{(}{)}{#1}{#2, #3}} % open interval
\newcommandx{\intvclop}[3][1=normal]{\braces{[}{)}{#1}{#2, #3}} % half-open interval (right)
\newcommandx{\intvopcl}[3][1=normal]{\braces{(}{]}{#1}{#2, #3}} % half-open interval (left)
\newcommand{\dotarg}{\ensuremath{\raisebox{.15ex}{$\scriptstyle [\cdot]$}}} % a dot, where a variable could be put into
\DeclareMathOperator*{\argmin}{argmin} % argmin
\DeclareMathOperator*{\argmax}{argmax} % argmax
\DeclareMathOperator{\sign}{sign}
\newcommandx{\abs}[2][1=normal]{\braces{\lvert}{\rvert}{#1}{#2}} % absolute value
\newcommand{\conj}[1]{\overline{#1}} % complex conjugation
\newcommandx{\ceil}[2][1=normal]{\braces{\lceil}{\rceil}{#1}{#2}} % ceil
\newcommandx{\floor}[2][1=normal]{\braces{\lfloor}{\rfloor}{#1}{#2}} % floor
\newcommandx{\round}[2][1=normal]{\braces{[}{]}{#1}{#2}} % round
\newcommandx{\der}[1]{D^{#1}} % differential operator (#1 = multiindex)
\newcommandx{\partder}[4][1={},4={}]{\frac{\partial^{#4} #2}{\partial #3^{#4}}\ifargdef{#1}{\Big|_{#1}}} % partial derivative (#1=point of evaluation, #2=function, #3=variable, #4=order)
\newcommandx{\integ}[4][1={},2={}]{\int_{#1}^{#2} #3 \, #4} % integral (#1op=lower bound, #2op=upper bound, #3=integrand, #4=differential form)
\newcommandx{\asympffaster}[2][1=normal]{o\braces{(}{)}{#1}{#2}} % asymptotically faster (proper) (#1op=size)
\newcommandx{\asympfaster}[2][1=normal]{O\braces{(}{)}{#1}{#2}} % asymptotically faster
\newcommandx{\asympeq}[2][1=normal]{\Theta\braces{(}{)}{#1}{#2}} % asymptotically equal
\newcommandx{\asympsslower}[2][1=normal]{\omega\braces{(}{)}{#1}{#2}} % asymptotically slower (proper)
\newcommandx{\asympslower}[2][1=normal]{\Omega\braces{(}{)}{#1}{#2}} % asymptotically slower

%%% --> Linear Algebra and Functional Analysis <--
\DeclareMathOperator{\Id}{Id} % identity operator
\newcommand{\GL}[2]{\operatorname{GL}(#1, #2)} % general linear group (#1=dimension, #2=field)
\newcommand{\matr}[1]{\begin{bmatrix} #1 \end{bmatrix}} % matrix
\newcommand{\smallmatr}[1]{\left[\begin{smallmatrix} #1 \end{smallmatrix}\right]} % small matrix
\newcommandx{\norm}[2][1=normal]{\braces{\|}{\|}{#1}{#2}} % norm
\renewcommandx{\sp}[3][1=normal]{\braces{\langle}{\rangle}{#1}{#2, #3}} % inner product (#1op=size, #2=left, #3=right)
\newcommand{\adj}[1]{{#1}^\ast} % adjoint operator
\newcommandx{\End}[2][2={}]{\mathcal{L}\opleft( #1 \ifargdef{#2}{, #2} \opright)} % endomorphism (#1=from, #2op=to)
\newcommand{\orthsum}{\oplus} % orthogonal sum
\newcommand{\orthcompl}[1]{{#1}^\perp} % orthogonal complement
\newcommand{\tensprod}{\otimes} % tensor product
\DeclareMathOperator{\ran}{ran} % image/range
% \DeclareMathOperator{\ker}{ker} % Kern
\DeclareMathOperator{\spann}{\operatorname{span}} % span
\newcommand{\T}{\top} % transposition (of a matrix)
\newcommand{\embeds}{\hookrightarrow} % embedding
\renewcommand{\vec}[1]{\boldsymbol{#1}} % vectors in boldface

%%% --> Function Spaces <--
% \newcommand{\almostev}{\text{f.ü.}} % fast überall
\newcommand{\almostev}{\text{a.e.}} % almost everywhere
\newcommandx{\measure}[2][1=normal]{\operatorname{vol}\braces{(}{)}{#1}{#2}} % Lebesgue-measure/volume of a set
\newcommand{\indset}[1]{\chi_{#1}} % indicator function for sets
\newcommand{\indcoeff}[1]{\mathds{1}_{#1}} % indicator function for sequences
\DeclareMathOperator{\supp}{supp} % support
\newcommandx{\Leb}[3][1={},3=normal]{L^{#2}\ifargdef{#1}{\braces{(}{)}{#3}{#1}}{}} % Lebesgue spaces (#1op=set, #2=exponent)
\newcommandx{\Lebnorm}[4][1=normal,3={2},4={}]{\norm[#1]{#2}_{#3}} % Lebesgue norm (#1op=size, #2=content, #3op=exponent, #4op=set)
\renewcommandx{\l}[3][1={},3=normal]{\ell^{#2}\ifargdef{#1}{\braces{(}{)}{#3}{#1}}} % lp sequence spaces (#1op=set, #2=exponent)
\newcommandx{\lnorm}[4][1=normal,3={2},4={}]{\norm[#1]{#2}_{#3}} % lp norm (#1op=size, #2=content, #3op=exponent, #4op=set)
\newcommandx{\Smooth}[4][1={},3={},4=normal]{C_{#3}^{#2}\ifargdef{#1}{\braces{(}{)}{#4}{#1}}} % space of differentiable functions (#1op=set, #2=order, #3op=modifier)
\newcommandx{\Schwartz}[2][1={},2=normal]{\mathscr{S}\ifargdef{#1}{\braces{(}{)}{#2}{#1}}} % space of Schwartz functions
\newcommandx{\Schwartzpoly}[2][1=normal]{\braces{\langle}{\rangle}{#1}{\abs[#1]{#2}} } % Schwartz polynomial
\newcommand{\Schwartznorm}[3]{C_{#1,#2}(#3)} % Schwartz space norm
\newcommandx{\Tempdistr}[2][1={},2=normal]{\mathscr{S}'\ifargdef{#1}{\braces{(}{)}{#2}{#1}}} % tempered distributions
\newcommandx{\distrinp}[3][1=normal]{\braces{\langle}{\rangle}{#1}{#2, #3}} % evaluation of a tempered distribution (#1op=size, #2=distribution, #3=Schwartz function)
\newcommand{\Linedistr}[1][]{\mathfrak{L}\ifargdef{#1}{_{#1}}{}} % line distribution
\newcommand{\conv}{\star} % convolution operator
\newcommandx{\ft}[3][1=default,2=auto]{
\ifstrequal{#1}{default}{\widehat{#3}}{
\ifstrequal{#1}{long}{{\braces{(}{)}{#2}{#3}}^{\wedge}}{}}} % Fourier transform (hat-notation) (#1op=long expression mode, #2op=size, #3=content)
\newcommand{\ftop}{\mathcal{F}} % Fourier transform (operator notation)
\newcommandx{\ift}[3][1=default,2=auto]{
\ifstrequal{#1}{default}{\check{#3}}{
\ifstrequal{#1}{long}{{\braces{(}{)}{#2}{#3}}^{\vee}}{}}} % inverse Fourier transform (hat-notation) (#1op=long expression mode, #2op=size, #3=content)
\newcommand{\iftop}{\mathcal{F}^{-1}} % inverse Fourier transform (operator notation)

%%% --> Miscellaneous <--
\newcommand{\sinc}{\operatorname{sinc}} % sinc function